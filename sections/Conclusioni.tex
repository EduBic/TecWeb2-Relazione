
\section{Conclusioni}
	Di seguito si trova la tabella che riassume le valutazioni date per ogni tematica analizzata del sito. I voti sono dati soggettivamente per questo sotto la colonna \textbf{note} si cerca di darne una motivazione e una giustificazione.
	

	\begin{table} [h]
		\centering
		\begin{tabularx}{\textwidth}{lXc}
			\toprule
			\textbf{Tematica} & \multicolumn{1}{c}{\textbf{Note}} & \textbf{Voto} \\
			\toprule
			Assi comunicativi & Buoni gli assi \textit{Why} e \textit{When}, quasi accettabili gli assi \textit{What} e \textit{Who}, male per gli assi \textit{Where} e \textit{How} & 6 \\
			\midrule
			Navigabilità & Insoddisfacente, navigare all'interno del sito spesso porta disorientamento all'utente a cui è richiesto lo sforzo della creazione di una mappa mentale poiché manca un breadcrump adeguato & 5 \\
			\midrule
			Contenuto & Buono il testo peccato che tutto il sito dia precedenza alle immagini, troppe e causa di frustrazioni per l'utente in molteplici casi. Discutibili alcune scelte di design a scapito dell'usabilità (link e metafore visive) & 5 \\
			\midrule
			Pubblicità & Discutibili alcune scelte ma comprensibili visti gli scopi del sito & 6.5 \\
			\midrule
			Ricerca & Punto più grave insieme alla mancanza di breadcrump. La cattiva gestione di essa riduce in modo drastico l'usabilità del sito  & 4.5 \\
			\midrule
			Responsiveness & Molto bene il design che supporta pienamente il mondo mobile, peccato per le prestazioni molto lente che penalizzano enormemente le visite del sito attraverso dispositivi mobili & 5.5 \\
			\midrule
			\textbf{Media} & & \textbf{5.4} \\
			\midrule
		\end{tabularx}
		\caption{Valutazioni finali sugli aspetti analizzati e la media aritmetica}
		\label{tab:ValutazioniFinali}
	\end{table}
	
	\subsection{Valutazione finale} 
		Considerato quanti siti al giorno d'oggi peccano nell'usabilità \href{http://www.cinisio.com/}{cinisio.com} cerca di arrivare alla sufficienza presentando scelte azzeccate ma purtroppo ancora troppe mancanze gravi. Non si può quindi affermare che il sito giunga ad una sufficienza o quasi sufficienza.
		
		\begin{center}
		\begin{tabular}{c|r}
			\textbf{Voto finale} & \textbf{5} \\
		\end{tabular}
		\end{center}
		