
\section{Analisi preliminare}

	\subsection{Note introduttive}
		Ogni figura del presente documento che riguarda il sito internet preso in analisi è prelevata dalle immagini allegate con il presente documento che illustrano completamente e senza modifiche le pagine del sito analizzate. Di seguito la lista di esse:
		\begin{itemize}
			\item \href{http://www.cinisio.com/}{Cinisio.com} (nel testo riferita come homepage);
			\item \href{http://www.cinisio.com/news}{Cinisio.com » News};
			\item \href{http://www.cinisio.com/racing/calendario}{Cinisio.com » Calendario};\\
			\item \href{http://www.cinisio.com/racing}{Cinisio.com » Cinisio Racing};
			\item \href{http://www.cinisio.com/contattami}{Cinisio.com » Contattami};
			\item \href{http://www.cinisio.com/2016/500-miglia-di-pomposa-unimpresa-quasi-riuscita/}{Cinisio.com » 500 miglia di Pomposa- arrivano i The Mello Yellos!} (nel testo indicata come pagina new o articolo);
			\item \href{}{Cinisio.com (Senza immagini)} (ottenuta tramite plugin disabilita immagini);
			\item \href{}{Cinisio.com (mobile)} (ottenuta tramite tool integrato nel browser);
		\end{itemize}

	\subsection{Il sito cinisio.com}
		Il sito preso in esame, \href{http://www.cinisio.com/}{www.cinisio.com}, contiene le informazioni inerenti ad un team di karting amatoriale. Esso ne racconta la storia attraverso un archivio di articoli che parlano delle gare intraprese dal team o di eventi esterni accaduti nell'ambito racing (Nascar, Formula 1 etc.). Tra i contenuti risultano esserci anche eventi sempre in ambito racing organizzati per associazioni di volontariato. Il sito mette a disposizione un calendario costantemente aggiornato con gli eventi passati e prossimi. Per tali motivi si presta molto ad essere identificato come un sito blog.
		
	\subsection{La struttura}
		Il sito presenta la seguente struttura gerarchica di pagine:
		\begin{itemize}
			\item Homepage;
				\begin{itemize}
					\item News: in cui vengono raccolti tutti gli articoli pubblicati;
						\begin{itemize}
							\item articolo 1;
							\item \dots
							\item articolo N;
						\end{itemize}
					\item Team: in cui vengono presentati il gruppo di amici che forma il team amatoriale di kart;
					\item Gare: un calendario di tutti gli eventi organizzati.
					\item Contatti: vengono presentate tutte le informazioni del proprietario e gestore del sito.
				\end{itemize}
		\end{itemize}
		
		La gerarchia è molto semplice, tuttavia, grazie al grande numero di anni di attività (la prima new del sito risale al 2011 mentre il dominio è proprietario dal 2008), il sito raggiunge una quote di centinaia di pagine, la maggioranza delle quali sono articoli, ossia pagine che possono mantenere valore anche a distanza di anni.
		
		