
\section{Assi di comunicazione}
	In questa sezione verrà discussa la comunicazione di informazioni da parte del sito verso l'utente. Nei confronti del sito l'utente si necessita di avere risposta alle seguenti domande, comunemente riconosciuti come assi principali della comunicazione: le 5 W +1:
	\begin{itemize}
		\item \textit{Who?} - Chi rappresenta il sito?
		\item \textit{What?} - Cosa offre il sito?
		\item \textit{When?} - Quali sono le ultime novità? Quand'è l'ultima volta che è stato manutenuto?
		\item \textit{Why?} - Perché mai dovrei fermarmi su questo sito? Quali benefici mi porta?
		\item \textit{Where?} - In quale punto del sito sono arrivato?
		\item \textit{How?} - Come faccio ad arrivare alle sezioni principali?
\end{itemize}	 

In questa sezione quindi proveremo a rispondere a queste 6 domande reperendo le informazioni che comunica all'utente il sito. 
	Le pagine analizzate saranno:
		\begin{itemize}
			\item homepage;
			\item una pagina di un articolo.
			%\item 
		\end{itemize}
		
		Per avere un riscontro più reale oltre all'autore della relazione si sono analizzate due esperienze di probabili utenti del sito. Ad entrambi sono state chieste le domande sopra elencate e di effettuare una specifica azione all'interno del sito ossia registrarsi tramite compilazione di un form all'evento \textit{Ac Kart Endurance 2016} organizzato partendo dalla homepage \ref{fig:Homepage_0}. Le pagine a cui si fa riferimento sono: l'\textit{homepage} e l'articolo \textit{500 miglia di Pomposa- arrivano i The Mello Yellos!} allegati alla presente relazione.
		
		\begin{figure} [h]
			\centering
			\includegraphics[width=\textwidth]{images/Homepage_0}
			\caption{Homepage}
			\label{fig:Homepage_0}
		\end{figure}
		

	\subsection{Who}
		
		
		\subsubsection{Homepage}
			La pagina risponde a tale domanda grazie al logo e alla presenza della pagina Team nel menu anche se in alcuni casi potrebbe non essere immediato e dal logo che contiene il nome del sito. Tuttavia per trovare gli interessati, le informazioni del team nella homepage, è necessario effettuare lo scroll per almeno due schermate, nel footer possiamo trovare delle fotografie che ritraggono i componenti del team. Più in basso ancora finalmente troviamo il nome e cognome dell'autore del sito. Nonostante le informazioni ci sono un po' sparse il risultato è insoddisfacente, l'utente deve risalire a tali informazioni solo aggiungendo sforzo: aprire la pagina Team o effettuare lo scroll. Inoltre le fotografie del team non sono cliccabili e quindi non si identificano i visi. Le informazioni dell'autore sono scarse, manca una sezione all'interno della pagina in cui l'autore o il team si descrivono.
		
		\subsubsection{Pagina di un articolo}
			Nella pagine di un articolo la situazione resta pressoché la stessa. Le informazioni sono sparse nel fondo della pagina e perché l'utente comprenda la figura dietro al sito è necessario lo sforzo di passare alla pagine \textit{Team} nel menu in alto.
			
		\subsubsection{Conclusioni}
			In entrambi i casi gli utenti a cui è stato chiesto chi rappresentasse il sito si sono fermati al logo e alle fotografie del team, scoraggiati nel reperire più informazioni senza l'aggiunta di sforzo. Nel complesso la comunicazione dell'asse viene valutata \textbf{sufficiente}.
	

	\subsection{What}
		
		\subsubsection{Homepage}
			Nella pagina risulta chiara l'offerta del sito: articoli su eventi del mondo motorsport. Il logo del sito e le immagini rispondono a tale domanda. Purtroppo però il sito non tratta soltanto di questi contenuti, gli articoli si suddividono in opinioni sul mondo dei motori, racconti delle ultime gare corse dal team, organizzazioni di eventi e altro ancora. La homepage del sito non comunica bene tutti i contenuti offerti dal sito peccando nella sua funzione principale.
		
		\subsubsection{Pagina di un articolo}
			Solitamente un utente interessato alle offerte si dirige alla homepage del sito. Purtroppo però il link alla home non è disponibile nel menu di navigazione costringendo all'uso ripetuto del pulsante back per chi l'aveva già visitata. In realtà il link esiste ed è l'immagine del logo, questo però non risulta essere intuitivo per tutti gli utenti. In alternativa un altro link \textit{Home} è situato nel footer della pagina, irraggiungibile da qualsiasi utente.
			
		\subsubsection{Conclusioni}
			Si è rilevato che l'homepage risponde a tale asse in modo approssimativo e non dà la giusta informazione per comprendere appieno i contenuti offerti dal sito. Inoltre tale asse risulta penalizzato dal fatto che non esista un pulsante \textit{Home} nel menu in alto, uno dei due utenti non ha mai cliccato il logo. Per tali motivi la comunicazione di tale asse è valutata \textbf{quasi sufficiente}.
		
	
	\subsection{When}
	
		\subsubsection{Homepage}
			L'homepage dovrebbe rispondere immediatamente a tale domanda grazie all'immagine a tutto schermo nella parte in alto che corrisponde all'ultimo articolo pubblicato. Tuttavia ciò non è stato rilevato dal comportamento dei soggetti in esame. Entrambi infatti hanno ignorato tale spazio occupato dall'immagine (didascalie comprese) e per trovare risposta a tale domanda sono finiti nella parte \textit{Racing News} qui ad ogni new (o articolo) è associata la data di pubblicazione. Fortunatamente l'articolo ultimo risulta nella lista a griglia anche in basso ripetendosi, se così non fosse stato un utente avrebbe facilmente confuso il penultimo articolo con l'ultimo. Un'altra dimostrazione di quanto le immagini siano di poco conto per gli utenti del web (per la questione immagini se ne discuterà in seguito).
		
		\subsubsection{Pagina di un articolo}
			Grazie alla presenza della parte \textit{Racing News} in tutte le pagine le informazioni sono facilmente reperibili. Tali link si ripetono anche sul lato destro della pagina evitando all'utente che non legge totalmente l'articolo di poter usufruirne comunque.
			
		\subsubsection{Conclusioni}
			Nel complesso le pagine rispondono a questa domanda grazie alla costante presenza della sottosezione \textit{Racing News} che seppur con qualche difetto (layout a griglia) dà all'utente le informazioni cercate. La comunicazione di tale asse è positiva e valutata come \textbf{buona}.
		
	\subsection*{Why}
	
		\subsubsection{Homepage}
			Grazie alla presenza nel logo della parola \textit{Racing} l'utente capisce immediatamente l'ambito e le tematiche trattate dal sito. Si potrebbe valutare l'inserimento di uno slogan per dare ancora più informazioni all'utente.
			
		\subsubsection{Pagina di un articolo}
			Come per la homepage la presenza della parola \textit{Racing} nel logo soddisfa la comunicazione di tale asse.
			
		\subsubsection{Conclusioni}
			Entrambi gli utenti sotto osservazione hanno risposto a tale domanda in modo immediato anche se come descritto nell'analisi dell'asse \textbf{What} tale informazioni non coprono totalmente i contenuti che il sito offre. Nel complesso la comunicazione di tale asse è valutata come \textbf{buona}.
			
		
	\subsection{Where}
	
		\subsubsection{Homepage}
			Lo stile landing page fa sì che un utente riconosca immediatamente se la pagina aperta è l'homepage del sito. Tuttavia nessuna zona della pagina segnala ciò.	
		
		\subsubsection{Pagina di un articolo}
			Anche nella pagina di un articolo non c'è nessun riferimento che esprima la posizione della pagina all'interno del sito e quindi la posizione dell'utente. Il breadcrump presente è solo quello di tipo attributo e risulta insufficiente (tale aspetto verrà approfondito in seguito) inoltre il menu di navigazione non segnala in nessun modo la pagina cliccata.
			
		\subsubsection{Conclusioni}
			Agli utenti presi in esame è stato chiesto di navigare all'interno del sito, in un caso l'utente spesso si è ritrovato disorientato e utilizzava il pulsante back per ritrovare una base solida da cui ripartire mentre nell'altro caso non ci sono stati problemi evidenti. Nel complesso la non presenza di un robusto breadcrump rende la comunicazione di tale asse \textbf{insufficiente}.
		
	\subsection{How}
		
		\subsubsection{Homepage}
			Il menu semplice di navigazione composto di sole 4 pagine garantisce all'utente l'accesso alle informazioni desiderate. Non risulta ottimale poiché la sezione chiamata \textit{News} in realtà è un archivio degli articoli e la sezione chiamata \textit{Gare} differisce con il titolo della pagina \textit{Calendario}. La funzionalità di ricerca è nascosta (nella homepage è raggiungibile tramite un link e non un'apposita search box), inoltre risulta non implementata (vedere la relativa sezione di approfondimento).
		
		\subsubsection{Pagina di un articolo}
			Nella pagina di un articolo abbiamo lo stesso menu della homepage mancante del link per la home (sostituito dal logo). Mentre per la funzionalità di ricerca ritroviamo un search box sul lato destro della pagina nascosto ma a differenza di quello in homepage funzionante.
			
		\subsubsection{Conclusioni}
			Per lo scorretta offerta della funzionalità di ricerca la comunicazione di tale asse ha reso difficoltosa la ricerca della pagina da cui iscriversi per l'evento \textit{Ac Kart Endurance 2016} esplicitamente richiesto. Uno dei due utenti frustrato dai troppi sforzi andati perduti dopo poco ha rinunciato. La valutazione risulta \textbf{non sufficiente}.
			
			
	\subsection{Tempi comunicativi}
		Durante i test si è cercato di misurare approssimativamente il tempo per reperire le informazioni. In tutti gli assi ad eccezione del where i tempi sono piuttosto buoni anche se le informazioni comunicate non siano pienamente soddisfacenti. Per l'asse where a causa degli errori prima descritti i tempi sono cresciuti enormemente.
		
		Poiché i tempi misurati sono approssimativi si è deciso di non tenerne conto nella valutazione finale.