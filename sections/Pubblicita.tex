
\section{Pubblicità}
	Il sito non presenta forme di pubblicità aggressiva (pop up o banner di grandi dimensioni) ma presenta solo dei piccoli banner classici di Google. Tuttavia a questi non gli è data importanza.
	
	\subsection{Banner}
	Nella homepage l'unico banner presente è situato nel pre-footer (figura \ref{fig:Homepage-Banner}) ed è sicuramente invisibile per la grande maggioranza dell'utenza.
	
	\begin{figure} [h]
		\includegraphics[width=\textwidth]{images/OpsABanner}
		\caption{L'unica pubblicità contenuta nella homepage}
		\label{fig:Homepage-Banner}
	\end{figure}
	
	In alcune pagine di new invece il banner entra in gioco ed è situato in primo piano a sinistra vicino al testo (figura \ref{fig:New-Banner}). Il posizionamento è azzeccato, non spezza il testo e non genera nessun fastidio per l'utente che può facilmente ignorarlo.
	
	
	In altre new e nelle altre pagine invece il banner è ancora messo in secondo piano, questa volta nella parte laterale destra in basso del sito. Anche qui è raro che un utente lo possa visualizzare.
	
	\subsection{Conclusioni}
		Da quanto visualizzato il sito non utilizza propriamente la pubblicità e dimostra quindi di non avere nessuno scopo di lucro. I posizionamenti anche se controproducenti per l'idea di banner non si sovrappongono mai all'usabilità del sito che rimane intatta.
	
	\begin{figure} [h]
		\includegraphics[width=\textwidth]{images/OpsABannerInANew}
		\caption{Posizione del banner in alcune pagine news}
		\label{fig:New-Banner}
	\end{figure}	
	
	\begin{figure} [h]
		\includegraphics[width=\textwidth]{images/OpsAnotherBanner}
		\caption{Posizione alternativa del banner in altre pagine del sito}
		\label{fig:AnotherBanner}
	\end{figure}